% exercise sheet with header on every page for math or close subjects
\documentclass[12pt]{article}
\usepackage[utf8]{inputenc} 
\usepackage{latexsym} 
\usepackage{multicol}
\usepackage{fancyhdr}
\usepackage{amsfonts} 
\usepackage{amsmath}
\usepackage{amssymb}
\usepackage{enumerate}
\usepackage{listings}
\usepackage{graphicx}

% Shortcuts for bb, frak and cal letters
\newcommand{\E}{\mathbb{E}}
\newcommand{\V}{\mathbb{V}}
\renewcommand{\P}{\mathbb{P}}
\newcommand{\N}{\mathbb{N}}
\newcommand{\R}{\mathbb{R}}
\newcommand{\C}{\mathbb{C}}
\newcommand{\Z}{\mathbb{Z}}
\newcommand{\Pfrak}{\mathfrak{P}}
\newcommand{\Pfrac}{\mathfrak{P}}
\newcommand{\Bfrac}{\mathfrak{P}}
\newcommand{\Bfrak}{\mathfrak{B}}
\newcommand{\Fcal}{\mathcal{F}}
\newcommand{\Ycal}{\mathcal{Y}}
\newcommand{\Bcal}{\mathcal{B}}
\newcommand{\Acal}{\mathcal{A}}

% formating
\topmargin -1.5cm 
\textheight 24cm
\textwidth 16.0 cm 
\oddsidemargin -0.1cm

% Fancy Header on every Page
\pagestyle{fancy}
\lhead{\textbf{Programmierung for Engineers - Exercise 1}}
\rhead{Daniel Schäfer (2549458)\\ Dominik Weber}
\renewcommand{\headrulewidth}{1.2pt}
\setlength{\headheight}{110pt} 

\begin{document}
\pagenumbering{gobble}
\lstset{language=C++}

\section{ Erste Schritte mit dem Arduino Mega}
\begin{enumerate}
    \item 
        ein weiteres \verb!delay(1000);! unter der Zeile \verb!digitalWrite(led , LOW);! einfuegen.
    \item
        \begin{enumerate}[a)]
            \item 
                see code:\\
                \begin{lstlisting}[frame=single]  % Start your code-block

                int led = 13;

                void setup () {
                    pinMode(led ,OUTPUT);
                }

                void loop () {
                    digitalWrite(led , HIGH);
                    delay (2500);
                    digitalWrite(led , LOW);
                    delay (2500);
                }
                \end{lstlisting}

            \item
                see code:\\
                \begin{lstlisting}[frame=single]  % Start your code-block

                int led = 13;

                void setup () {
                    pinMode(led ,OUTPUT);
                }

                void loop () {
                    digitalWrite(led , HIGH);
                    delay (3000);
                    digitalWrite(led , LOW);
                    delay (1000);
                }
                \end{lstlisting}


            \newpage
            \item
                see code:\\
                \begin{lstlisting}[frame=single]  % Start your code-block

                int led = 13;

                void setup () {
                    pinMode(led ,OUTPUT);
                }

                void loop () {
                    digitalWrite(led , HIGH);
                    delay (500);
                    digitalWrite(led , LOW);
                    delay (500);
                }
                \end{lstlisting}
                
        \end{enumerate}
\end{enumerate}


    \newpage
\section{Morse}

\begin{enumerate}
    \item
        see file \verb!Morse.ino!
    \item
        with the used approach this would be very messy! Essentially you rewrite the approach by using an int argument called \verb!LED_id! to all functions called \verb!morse_A(int LED_id)!, \verb!morse_B(int LED_id)! and so on. Instead of calling \verb! dit()! and \verb!dah()! we would call \verb!dit(int LED_id)! and \verb!dah(int LED_id)!. The functions would look like this:\\

        \begin{lstlisting}[frame=single]  % Start your code-block

        void dit(int LED_id) {
            Serial.print(".");

            // send a dit
            digitalWrite(LED_id, HIGH);
            delay(dit_delay);

            digitalWrite(LED_id, LOW);
            delay(dit_delay);
        }

        void dah(int LED_id) {
            Serial.print("-");
          
            // send a dah
            digitalWrite(LED_id, HIGH);
            delay(dah_delay);

            digitalWrite(LED_id, LOW);
            delay(dit_delay);
        }

        \end{lstlisting}

        now you can just call the function with the id of the LED you want the letter to appear as shown in the file for this task. You can replace the LED id 13 with any other LED if you would like!

\end{enumerate}


\end{document}
